% Options for packages loaded elsewhere
\PassOptionsToPackage{unicode}{hyperref}
\PassOptionsToPackage{hyphens}{url}
%
\documentclass[
]{article}
\usepackage{lmodern}
\usepackage{amssymb,amsmath}
\usepackage{ifxetex,ifluatex}
\ifnum 0\ifxetex 1\fi\ifluatex 1\fi=0 % if pdftex
  \usepackage[T1]{fontenc}
  \usepackage[utf8]{inputenc}
  \usepackage{textcomp} % provide euro and other symbols
\else % if luatex or xetex
  \usepackage{unicode-math}
  \defaultfontfeatures{Scale=MatchLowercase}
  \defaultfontfeatures[\rmfamily]{Ligatures=TeX,Scale=1}
\fi
% Use upquote if available, for straight quotes in verbatim environments
\IfFileExists{upquote.sty}{\usepackage{upquote}}{}
\IfFileExists{microtype.sty}{% use microtype if available
  \usepackage[]{microtype}
  \UseMicrotypeSet[protrusion]{basicmath} % disable protrusion for tt fonts
}{}
\makeatletter
\@ifundefined{KOMAClassName}{% if non-KOMA class
  \IfFileExists{parskip.sty}{%
    \usepackage{parskip}
  }{% else
    \setlength{\parindent}{0pt}
    \setlength{\parskip}{6pt plus 2pt minus 1pt}}
}{% if KOMA class
  \KOMAoptions{parskip=half}}
\makeatother
\usepackage{xcolor}
\IfFileExists{xurl.sty}{\usepackage{xurl}}{} % add URL line breaks if available
\IfFileExists{bookmark.sty}{\usepackage{bookmark}}{\usepackage{hyperref}}
\hypersetup{
  pdftitle={Class 6 Homework},
  pdfauthor={Sarra Larif},
  hidelinks,
  pdfcreator={LaTeX via pandoc}}
\urlstyle{same} % disable monospaced font for URLs
\usepackage[margin=1in]{geometry}
\usepackage{color}
\usepackage{fancyvrb}
\newcommand{\VerbBar}{|}
\newcommand{\VERB}{\Verb[commandchars=\\\{\}]}
\DefineVerbatimEnvironment{Highlighting}{Verbatim}{commandchars=\\\{\}}
% Add ',fontsize=\small' for more characters per line
\usepackage{framed}
\definecolor{shadecolor}{RGB}{248,248,248}
\newenvironment{Shaded}{\begin{snugshade}}{\end{snugshade}}
\newcommand{\AlertTok}[1]{\textcolor[rgb]{0.94,0.16,0.16}{#1}}
\newcommand{\AnnotationTok}[1]{\textcolor[rgb]{0.56,0.35,0.01}{\textbf{\textit{#1}}}}
\newcommand{\AttributeTok}[1]{\textcolor[rgb]{0.77,0.63,0.00}{#1}}
\newcommand{\BaseNTok}[1]{\textcolor[rgb]{0.00,0.00,0.81}{#1}}
\newcommand{\BuiltInTok}[1]{#1}
\newcommand{\CharTok}[1]{\textcolor[rgb]{0.31,0.60,0.02}{#1}}
\newcommand{\CommentTok}[1]{\textcolor[rgb]{0.56,0.35,0.01}{\textit{#1}}}
\newcommand{\CommentVarTok}[1]{\textcolor[rgb]{0.56,0.35,0.01}{\textbf{\textit{#1}}}}
\newcommand{\ConstantTok}[1]{\textcolor[rgb]{0.00,0.00,0.00}{#1}}
\newcommand{\ControlFlowTok}[1]{\textcolor[rgb]{0.13,0.29,0.53}{\textbf{#1}}}
\newcommand{\DataTypeTok}[1]{\textcolor[rgb]{0.13,0.29,0.53}{#1}}
\newcommand{\DecValTok}[1]{\textcolor[rgb]{0.00,0.00,0.81}{#1}}
\newcommand{\DocumentationTok}[1]{\textcolor[rgb]{0.56,0.35,0.01}{\textbf{\textit{#1}}}}
\newcommand{\ErrorTok}[1]{\textcolor[rgb]{0.64,0.00,0.00}{\textbf{#1}}}
\newcommand{\ExtensionTok}[1]{#1}
\newcommand{\FloatTok}[1]{\textcolor[rgb]{0.00,0.00,0.81}{#1}}
\newcommand{\FunctionTok}[1]{\textcolor[rgb]{0.00,0.00,0.00}{#1}}
\newcommand{\ImportTok}[1]{#1}
\newcommand{\InformationTok}[1]{\textcolor[rgb]{0.56,0.35,0.01}{\textbf{\textit{#1}}}}
\newcommand{\KeywordTok}[1]{\textcolor[rgb]{0.13,0.29,0.53}{\textbf{#1}}}
\newcommand{\NormalTok}[1]{#1}
\newcommand{\OperatorTok}[1]{\textcolor[rgb]{0.81,0.36,0.00}{\textbf{#1}}}
\newcommand{\OtherTok}[1]{\textcolor[rgb]{0.56,0.35,0.01}{#1}}
\newcommand{\PreprocessorTok}[1]{\textcolor[rgb]{0.56,0.35,0.01}{\textit{#1}}}
\newcommand{\RegionMarkerTok}[1]{#1}
\newcommand{\SpecialCharTok}[1]{\textcolor[rgb]{0.00,0.00,0.00}{#1}}
\newcommand{\SpecialStringTok}[1]{\textcolor[rgb]{0.31,0.60,0.02}{#1}}
\newcommand{\StringTok}[1]{\textcolor[rgb]{0.31,0.60,0.02}{#1}}
\newcommand{\VariableTok}[1]{\textcolor[rgb]{0.00,0.00,0.00}{#1}}
\newcommand{\VerbatimStringTok}[1]{\textcolor[rgb]{0.31,0.60,0.02}{#1}}
\newcommand{\WarningTok}[1]{\textcolor[rgb]{0.56,0.35,0.01}{\textbf{\textit{#1}}}}
\usepackage{graphicx,grffile}
\makeatletter
\def\maxwidth{\ifdim\Gin@nat@width>\linewidth\linewidth\else\Gin@nat@width\fi}
\def\maxheight{\ifdim\Gin@nat@height>\textheight\textheight\else\Gin@nat@height\fi}
\makeatother
% Scale images if necessary, so that they will not overflow the page
% margins by default, and it is still possible to overwrite the defaults
% using explicit options in \includegraphics[width, height, ...]{}
\setkeys{Gin}{width=\maxwidth,height=\maxheight,keepaspectratio}
% Set default figure placement to htbp
\makeatletter
\def\fps@figure{htbp}
\makeatother
\setlength{\emergencystretch}{3em} % prevent overfull lines
\providecommand{\tightlist}{%
  \setlength{\itemsep}{0pt}\setlength{\parskip}{0pt}}
\setcounter{secnumdepth}{-\maxdimen} % remove section numbering

\title{Class 6 Homework}
\author{Sarra Larif}
\date{1/23/2020}

\begin{document}
\maketitle

Improve this code: library(bio3d) s1 \textless- read.pdb(``4AKE'') \#
kinase with drug s2 \textless- read.pdb(``1AKE'') \# kinase no drug s3
\textless- read.pdb(``1E4Y'') \# kinase with drug s1.chainA \textless-
trim.pdb(s1, chain=``A'', elety=``CA'') s2.chainA \textless-
trim.pdb(s2, chain=``A'', elety=``CA'') s3.chainA \textless-
trim.pdb(s1, chain=``A'', elety=``CA'') s1.b \textless-
s1.chainA\(atom\)b s2.b \textless- s2.chainA\(atom\)b s3.b \textless-
s3.chainA\(atom\)b plotb3(s1.b, sse=s1.chainA, typ=``l'',
ylab=``Bfactor'') plotb3(s2.b, sse=s2.chainA, typ=``l'',
ylab=``Bfactor'') plotb3(s3.b, sse=s3.chainA, typ=``l'',
ylab=``Bfactor'')

\begin{Shaded}
\begin{Highlighting}[]
\NormalTok{plot.prot <-}\StringTok{ }\ControlFlowTok{function}\NormalTok{(PDB_ID, }\DataTypeTok{chain =} \StringTok{"A"}\NormalTok{, }\DataTypeTok{atom =} \StringTok{"CA"}\NormalTok{, }\DataTypeTok{sse =}\NormalTok{ b, }\DataTypeTok{typ =} \StringTok{"l"}\NormalTok{, }\DataTypeTok{ylab =} \StringTok{"Bfactor"}\NormalTok{)\{}
  \KeywordTok{library}\NormalTok{(bio3d)}
\NormalTok{  a <-}\StringTok{ }\KeywordTok{read.pdb}\NormalTok{(PDB_ID) }
\NormalTok{  b <-}\StringTok{ }\KeywordTok{trim.pdb}\NormalTok{(a, }\DataTypeTok{chain =}\NormalTok{ chain, }\DataTypeTok{elety =}\NormalTok{ atom)}
\NormalTok{  c <-}\StringTok{ }\NormalTok{b}\OperatorTok{$}\NormalTok{atom}\OperatorTok{$}\NormalTok{b}
  \KeywordTok{plotb3}\NormalTok{(c, }\DataTypeTok{sse =}\NormalTok{ sse, }\DataTypeTok{typ =}\NormalTok{ typ, }\DataTypeTok{ylab =}\NormalTok{ ylab)}
\NormalTok{\}}
\end{Highlighting}
\end{Shaded}

The inputs of this function include the PDB ID to identify the protein
of interest, the chain which is an input for the \texttt{trim.pdb} which
identifies the chain of interest in the protein, the atom type which is
normally the optional \texttt{elety} input in the \texttt{trim.pdb}, and
\texttt{sse}, \texttt{typ}, and \texttt{ylab} which specify the
appearance of the plot as well as its y-axis label. All the inputs in my
function except the protein ID have the same defaults as the original
provided code but still allows the user to make changes, should they
want to have different specifications for their protein or plot.
Therefore, all one really needs to do to use this function is enter the
PDB ID and optional parameters to receive an output of a scatter plot of
the selected protein structure residues including secondary structure
element representations in the margins. It does this by first using the
given PDB ID to read its data with \texttt{read.pdb}, then it narrows
the required data with \texttt{trim.pdb} with the optional inputs or
defaults and then plots said data using \texttt{plotb3}.

Examples of output:

\begin{Shaded}
\begin{Highlighting}[]
\KeywordTok{plot.prot}\NormalTok{(}\StringTok{"4AKE"}\NormalTok{)}
\end{Highlighting}
\end{Shaded}

\begin{verbatim}
##   Note: Accessing on-line PDB file
\end{verbatim}

\includegraphics{class06_hw_files/figure-latex/unnamed-chunk-2-1.pdf}

\begin{Shaded}
\begin{Highlighting}[]
\KeywordTok{plot.prot}\NormalTok{(}\StringTok{"2HHB"}\NormalTok{)}
\end{Highlighting}
\end{Shaded}

\begin{verbatim}
##   Note: Accessing on-line PDB file
\end{verbatim}

\includegraphics{class06_hw_files/figure-latex/unnamed-chunk-2-2.pdf}

\begin{Shaded}
\begin{Highlighting}[]
\KeywordTok{plot.prot}\NormalTok{(}\StringTok{"1E4Y"}\NormalTok{)}
\end{Highlighting}
\end{Shaded}

\begin{verbatim}
##   Note: Accessing on-line PDB file
\end{verbatim}

\includegraphics{class06_hw_files/figure-latex/unnamed-chunk-2-3.pdf}

\end{document}
